\newpage
%%%%%%%%%%%%%%%%%%%%%%%%%%%%%%%%%%%%%%%%%%%%%%%%%%%%%%%%%%%%%%%%
%%%%%%%%%%%%%%%%%%%%%%%%%%%%%%%%%%%%%%%%%%%%%%%%%%%%%%%%%%%%%%%%
%%%%%%%%%%%%%%%%%%%%%%%%%% Enunciado %%%%%%%%%%%%%%%%%%%%%%%%%%%

\begin{myblock}
\phantomsection\addcontentsline{toc}{section}{Ejercicio \#1 | Broadcasting}
\section*{Ejercicio \#1 | Broadcasting}

Calcula lo siguiente:

\[
    \begin{pmatrix} 1 & 2 \\ 3 & 4 \end{pmatrix} + 
    \begin{pmatrix} 0 & 1 \\ 2 & 3 \end{pmatrix} + 
    \begin{pmatrix} 7 & 9 \end{pmatrix}
\]

Usa broadcasting de tal forma que la operación esté bien definida. Antes, 
averigua y describe qué es broadcasting, en el contexto de numpy. 

\end{myblock}

%%%%%%%%%%%%%%%%%%%%%%%%%%%%%%%%%%%%%%%%%%%%%%%%%%%%%%%%%%%%%%%%
%%%%%%%%%%%%%%%%%%%%%%%%%%%%%%%%%%%%%%%%%%%%%%%%%%%%%%%%%%%%%%%%

%%%%%%%%%%%%%%%%%%%%%%%%%%%%%%%%%%%%%%%%%%%%%%%%%%%%%%%%%%%%%%%%
%%%%%%%%%%%%%%%%%%%%%%%%%%%%%%%%%%%%%%%%%%%%%%%%%%%%%%%%%%%%%%%%

\subsection{Teoría}

\subsection{Resumen de código}

\subsection{Resultados}

\subsection{Conclusiones}

\clearpage





















