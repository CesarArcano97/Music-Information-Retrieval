\newpage
%%%%%%%%%%%%%%%%%%%%%%%%%%%%%%%%%%%%%%%%%%%%%%%%%%%%%%%%%%%%%%%%
%%%%%%%%%%%%%%%%%%%%%%%%%%%%%%%%%%%%%%%%%%%%%%%%%%%%%%%%%%%%%%%%
%%%%%%%%%%%%%%%%%%%%%%%%%% Enunciado %%%%%%%%%%%%%%%%%%%%%%%%%%%

\begin{myblock}
\phantomsection\addcontentsline{toc}{section}{``Una nota personal sobre m´usica, sonido y
electrónica''}
\section*{``Una nota personal sobre m´usica, sonido y
electrónica''}

Haz un resumen corto del ensayo: \textit{``Una nota personal sobre música, sonido y electrónica''}

\end{myblock}

%%%%%%%%%%%%%%%%%%%%%%%%%%%%%%%%%%%%%%%%%%%%%%%%%%%%%%%%%%%%%%%%
%%%%%%%%%%%%%%%%%%%%%%%%%%%%%%%%%%%%%%%%%%%%%%%%%%%%%%%%%%%%%%%%

El ensayo de Daphne Oram comienza con una pregunta: ``¿Qué es la memoria?'' Es justo esa cuestión la
que abre el camino para que Oram presente su perspectiva sobre cómo es que funciona nuestra mente y, 
valga la redundancia, nuestra memoria. Y es que para Oram no es improbable que nuestra memoria funcione
como una especie de máquina reproductora de sonido (o de señales), al estilo de una grabadora de cintas. 

Las analogías que hace Oram tanto para explicar su idea de las memorias como el comportamiento humano se
centran en la música. Para ella, hay un proceso de retroalimentación que impulsa la resonancia de ciertas
memorias y secciones del cerebro. Uno de los primeros ejemplos que pone es el del estado de ánimo por las 
mañanas, argumentando que cuando estamos de buen humor los problemas parecen más livianos, mientras que
cuando estamos de malas, todo parece peor de lo que en realidad es. Menciona que: ``los sonidos musicales
que se reproducen a través del proceso de retroalimentación por cinta cambian de la misma manera; es decir, 
cambian de acuerdo a los ajustes de control de tono (o filtros) en el circuito de retroalimentación''. 

Interpreto lo anterior como que aquellas máquinas que procesan el sonido de los instrumentos, como un pedal 
para una guitarra eléctrica, o incluso la afinación de un piano o cualquier otro instrumento, determinan
la calidad del sonido y cómo el escucha los puede apreciar. Un violín mal afinado logrará desconcertar al 
público, así como un pedal llevado al límite puede producir sonidos no muy agradables para todos. La analogía 
es clara: los estados mentales son como la afinación o procesamiento de nuestros pensamientos, lo cual afecta 
directamente cómo nos sentimos a lo largo del día y cómo interpretamos los problemas. 

Oram también menciona que la grabación de recuerdos más pura y duradera es la de los recuerdos de la infancia.
Según ella, ``fueron grabados en una cinta virgen''. Tanto recuerdos buenos como malos en aquella etapa de la vida
seguramente perdurarán a lo largo del tiempo porque les pusimos especial atención o porque fueron tan intensos que
quedaron ``sobrecargados cuando se grabaron''. 

Bajo su óptica, no es como que la memoria sea una biblioteca en la cual existan cintas con los recuerdos grabados (esto
refleja un poco de la época en la que se escribió el ensayo). Para Oram, la memoria es como un circuito de retroalimentación
en el cual se emiten ecos continuos de señales hasta que se acaba la energía. Esto provoca que, cuando estemos recordando algo,
haya resonancia de tal manera que se vayan alimentando ciertas partes del circuito relacionadas con ciertos temas
(como si se tratara de tópicos). Interpreto que esto es algo así: si comienzo a pensar en alguien de mi pasado, quizás
recuerde con nostalgia, felicidad, enojo o tristeza a esa figura. Dependiendo de mi afinación inicial (estado de ánimo) y
del recuerdo en particular que esté proyectando, se generará una señal que probablemente alimentará o destruirá otras
señales. Se pueden comenzar a activar otras zonas del circuito que se relacionan con esa persona, como los lugares 
a los que fuimos, las emociones que se activaron en el recuerdo, incluso cosas como recuerdos de temperatura, olores,
o quizás hasta sonidos como tal.

\begin{quoting}
    Al clasificar las señales de la memoria, los mismos circuitos no se moverían de un área a otra en el cerebro, sino 
    que la energía, por mera inducción simpática, se encontraría resonando con células de retroalimentación del mismo tipo 
    en paralelo con aquellas ya ocupadas con el mismo tipo de información. Entonces, un pensamiento de memoria podría 
    resonar en muchas células de retroalimentación en múltiples partes del cerebro porque tiene afinidad con los pensamientos
    almacenados en esas regiones.
\end{quoting}

Dependiendo de la intensidad de la señal, que Oram llama algo así como ``retroalimentación de la memoria'', se 
provocarán resonancias que hagan crecer otros tópicos y comiencen a crecer en magnitud. Esto nos lleva a algo que Oram
aborda casi al final de su ensayo: los estados mentales que nos hunden en ansiedad, depresión o cosas similares (aunque
Oram no menciona dichas definiciones, creo que es claro que se refiere a ello). 

Además, también argumenta que la memoria no es perfecta. Oram argumenta que el cerebro probablemente prefiere no esforzarse
por recordar todo a detalle, sino que hay una parte del mismo que se encarga de ``generar'' ciertos detalles, a la cual
le llama: ``departamento de racionalización''. Creo que esto se relaciona un poco con la IA generativa, pues no es que
los algoritmos generativos se encarguen de entender y procesar lo que van a crear, sino que calculan lo más probable. 
Es decir, proyectan para los videos, los textos, las imágenes o los sonidos lo que es más probable que siga después de 
su último estado. Así, el cerebro puede quedarse con una proyección en baja dimensión de los recuerdos, y de esa manera
rellenar los huecos con lo que es más probable que sucedió bajo ese contexto. 

Quizás el último cuarto del ensayo es el que más me interesó. En esta sección, Oram comienza hablando acerca de los 
``estados de retroalimentación sin control''. Menciona que el cerebro humano, bajo la óptica de que se puede describir como
una combinación de señales de sonido, puede colapsar si no se controla apropiadamente el volumen de reproducción. ``Esta sobrecarga
que se da por subir demasiado el volumen de la reproducción ciertamente es algo en lo que vale la pena reflexionar: parece ser la 
perdición de nuestra existencia''. Oram continúa diciendo que si insistimos en subir demasiado el volumen, los problemas (pensamientos)
que estemos amplificando solo se volverán más grandes en nuestra mente: ``cada simple onda sinusoidal termina convirtiéndose en ruido
blanco''. Al permitir la resonancia de los pensamientos negativos sin control alguno, terminaremos convirtiéndolos en una especie de ``muro
de sonido'', pero no en el buen sentido, sino en uno que nos abruma, perturba y nos deja en un estado mental peor al cual en
el que iniciamos, ``entre más te concentres en esos pensamientos, más empeoran''.

Para Oram es claro que en el caso de un reproductor musical es tan sencillo como bajar el volumen. Sin embargo, también tiene claro
que el cerebro y la mente humana son mucho más complejos que eso. No basta con intentar ``bajar el volumen'' a la ansiedad, a los pensamientos
intrusivos. Una primera aproximación es llevar el discurso interno a otro tema, ``con la esperanza de que ese nuevo tema no sobrecargue 
el sistema de grabación con los mismos terribles resultados que antes''. 

Una segunda opción es la terapia de choque. Si dejamos que el cerebro se sobrecargue, seguramente en algún momento la misma intensidad
de los pensamientos nos lleve a la eliminación repentina y violenta de esa sensación negativa que nos provoca. Me pregunto si Oram
se estaba refiriendo quizá a la despersonalización o disociación de la mente con respecto a esos problemas. Quiero entender que para 
Oram esta no era la mejor opción, pues llevar al límite la mente puede resultar muy mal, no solo a nivel personal, sino también social. 
Llevarse al límite puede terminar en las peores decisiones. 

La tercera opción, y la que para Oram es todo un alivio, es un ``sencillo remedio al que me refiero: cuando un pensamiento te 
preocupa demasiado, ve y díselo a alguien, o escríbelo en una carta. Tal vez la razón por la que esto es un remedio es que el 
pensamiento pierde su energía cuando se transduce a la palabra hablada o escrita''. Sacar lo que lleva cargando tu mente siempre
funciona como un regulador de presión, una válvula de escape que desinflama un poco el dolor o pesadez con la cual vamos andando
por el mundo, mental y físico. 

Para Oram esta tercera opción es maravillosa porque los pensamientos se someten a una especie de censura que filtra las sensaciones. Además de que
cambia la perspectiva y ``el patrón de onda personal'' que eras cuando lo pensabas es distinto al que eres cuando lo compartes, lo
hablas o lo escribes. Puede interpretarse como ver con mayor perspectiva las cosas, o como una intuición musical: ``si escuchamos veinte 
violines interpretando la misma frase musical al unísono [...] el sonido resultante tiene una cualidad placentera que no está presente
cuando la misma frase musical, diseñada para veinte intérpretes, se toca por un mero violín''. Oram menciona que un solo para violín
puede ser muy rígido, pero con varios acompañándolo, su sonido se vuelve más acogedor. 

Creo que puedo interpretar esto también como una especie de cambio de dimensionalidad. Imagino los pensamientos como existiendo en alta 
dimensionalidad: nos afectan a nivel físico, anímico y mental. Si dejamos habitando solo en nuestra conciencia a los pensamientos intrusivos y negativos, 
y no los procesamos de alguna manera, terminarán envolviéndonos de manera casi invisible. Sin embargo, al traducirlos a algo como la escritura, los 
bajamos a una dimensión más pequeña, los encapsulamos en algo casi unidimensional como lo son las palabras plasmadas en una pantalla o en una hoja
de papel. Cuando son palabras escritas, no nos pueden hacer daño a menos que los dejemos volver a la mente de inmediato. Si los hablamos con 
alguna otra persona, también reducimos su dimensión, y aún mejor que con lo escrito, pues las palabras no perduran en el espacio, se difuminan casi 
al instante de decirlas, y el acompañamiento de la familia o los amigos siempre nos dará un confort aún más grande. Entonces, reducir la dimensión de
los pensamientos y quedarse con las características más importantes nos ayuda a procesarlos para vivir un día a día un poco más tranquilo. 

Como alguien ansioso y acostumbrado a una tormenta de pensamientos constante, encontré esta lectura muy liberadora. Justo me ayudó a ver con otra
óptica las cosas y fue una descarga para pensar en que se puede salir de los ciclos de retroalimentación negativa. Investigué un poco sobre Daphne Oram; 
es muy interesante saber que fue una de las primeras en introducir la música electrónica en Reino Unido y que creó una máquina llamada Oramics para traducir
figuras trazadas a sonido. Sin duda debió haber sido una persona muy interesante de conocer.



%%%%%%%%%%%%%%%%%%%%%%%%%%%%%%%%%%%%%%%%%%%%%%%%%%%%%%%%%%%%%%%%
%%%%%%%%%%%%%%%%%%%%%%%%%%%%%%%%%%%%%%%%%%%%%%%%%%%%%%%%%%%%%%%%

\clearpage