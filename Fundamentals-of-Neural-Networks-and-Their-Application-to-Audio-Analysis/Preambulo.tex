\usepackage{mathtools}
\usepackage{physics}
\usepackage[utf8]{inputenc}
\usepackage[spanish]{babel}
\usepackage{amsmath}
\usepackage{amssymb}
\usepackage{fancyhdr}
\setlength{\headheight}{14.5pt}
\decimalpoint
\usepackage{listings}
\usepackage{multicol}
\usepackage{tensor}
\usepackage{cancel}
\usepackage{sidecap}
\usepackage{setspace}
\usepackage{subcaption}
\usepackage{stix}
\usepackage[hidelinks]{hyperref}
\usepackage{tikz}
\usepackage{xcolor}
\usepackage{pgfplots}
\usepackage{cancel}
\usepackage{geometry}
\usepackage{empheq}
\pgfplotsset{width=10cm,compat=1.9}
\usepackage[T1]{fontenc}
\usepackage{titlesec, blindtext, color}
\definecolor{gray75}{gray}{0.75}
\newcommand{\hsp}{\hspace{20pt}}
\titleformat{\chapter}[hang]{\Huge\bfseries}{ \thechapter\hsp\textcolor{gray75}{|}\hsp}{0pt}{\Huge\bfseries}
\usepackage{tocloft}
\renewcommand{\cftsecleader}{\cftdotfill{\cftdotsep}}
\sidecaptionvpos{figure}{tl}

%%%%%%%%%%%%%%%%%%%%%%%%%%%%%%%%%%%%%%%%%
%%%%%      Underline Eqs        %%%%%%%%%
%%%%%%%%%%%%%%%%%%%%%%%%%%%%%%%%%%%%%%%%%

\newsavebox\MBox
\newcommand\Cline[2][red]{{\sbox\MBox{$#2$}%
  \rlap{\usebox\MBox}\color{#1}\rule[-1.2\dp\MBox]{\wd\MBox}{0.5pt}}}
%%%%%%%%%%%%%%%%%%%%%%%%%%%%%%%%%%%%%%%%%%
%%%        CONSTANTES                %%%%%
%%%%%%%%%%%%%%%%%%%%%%%%%%%%%%%%%%%%%%%%%%
\newcommand\mom{\sqrt{\frac{m\omega}{2\hbar}}}
\newcommand\momsq{\frac{m\omega}{2\hbar}}
\newcommand\dad{\sqrt{\frac{m\omega\hbar}{2}}ie^{-\abs{z}^2/2}}
\newcommand\gus{\sqrt{\frac{\hbar}{2m\omega}}}
\newcommand\gussq{\frac{\hbar}{2m\omega}}
\newcommand\adag{\hat{a}^{\dagger}}
\newcommand\ah{\hat{a}}

\newcommand{\ylm}[2]{Y_{#1}^{#2}}
\newcommand{\fun}[3]{\Psi_{#1 #2 #3}(r,\theta,\phi)=R_{#1 #2}(r)Y_{#2}^{#3}(\theta,\phi)}
\newcommand{\funup}[3]{\Psi_{#1 #2 #3,\,+\frac{1}{2}}(r,\theta,\phi)=R_{#1 #2}(r)Y_{#2}^{#3}(\theta,\phi)}
\newcommand{\fundown}[3]{\Psi_{#1 #2 #3,\,-\frac{1}{2}}(r,\theta,\phi)=R_{#1 #2}(r)Y_{#2}^{#3}(\theta,\phi)}
%%%%%%%%%%%%%%%%%%%%%%%%%%%%%%%%%%%%%%%%%%%
%%%%     PALETA DE COLORES         %%%%%%%%
%%%%%%%%%%%%%%%%%%%%%%%%%%%%%%%%%%%%%%%%%%%
\definecolor{Verde}{HTML}{31B189}
\definecolor{Azul}{HTML}{7ACE67}
\definecolor{Naranja}{HTML}{FFC872}
\definecolor{Crema}{HTML}{FFE3B3}
\definecolor{AzulUV}{HTML}{1C36A1}
\definecolor{VerdeUV}{HTML}{00AA33}
\definecolor{RCimat}{HTML}{73243D}
\definecolor{GCimat}{HTML}{ACAAAB}




%%%%%%%%%%%%%%%%%%%%%%%%%%%%%%%%%%%%%%%%%%%%%%
%%%        Configuración bloques grises    %%%
%%%%%%%%%%%%%%%%%%%%%%%%%%%%%%%%%%%%%%%%%%%%%%
\usetikzlibrary{positioning, shapes}
\pgfdeclarelayer{background}
\pgfdeclarelayer{foreground}
\pgfsetlayers{background,main,foreground}
\usepackage{tcolorbox}
\tcbuselibrary{skins,breakable}
\usetikzlibrary{shadings,shadows}
\newenvironment{myblock}[1]{%
    \tcolorbox[beamer,%
    noparskip,breakable,
    colback=white,colframe=RCimat,%
    colbacklower=white!75!RCimat,%
    title=#1,
    coltitle = black]}%
    {\endtcolorbox}


    
\newtcbox{\othermathbox}[1][]{nobeforeafter, tcbox raise base, enhanced, sharp corners, colback=white!10, colframe=AzulUV!30!black,drop fuzzy shadow, left=1em, top=1em, right=1em, bottom=1em}

% Syntax: \colorboxed[<color model>]{<color specification>}{<math formula>}
\newcommand*{\colorboxed}{}
\def\colorboxed#1#{%
  \colorboxedAux{#1}%
}
\newcommand*{\colorboxedAux}[3]{%
  % #1: optional argument for color model
  % #2: color specification
  % #3: formula
  \begingroup
    \colorlet{cb@saved}{.}%
    \color#1{#2}%
    \boxed{%
      \color{cb@saved}%
      #3%
    }%
  \endgroup
}



%%%%%%%%%%%%%%%%%%%%%%%%%%%%%%%%%%%%%%%%%%%%%%
%%%          Para poner código en R        %%%
%%%%%%%%%%%%%%%%%%%%%%%%%%%%%%%%%%%%%%%%%%%%%%
\lstset{language=R,
    basicstyle=\fontsize{9}{10}\selectfont\ttfamily,
    stringstyle=\color{purple},
    otherkeywords={0,1,2,3,4,5,6,7,8,9},
    morekeywords={TRUE,FALSE},
    deletekeywords={data,frame,length,as,character},
    keywordstyle=\color{blue},
    commentstyle=\color{purple},
    showstringspaces=false
}